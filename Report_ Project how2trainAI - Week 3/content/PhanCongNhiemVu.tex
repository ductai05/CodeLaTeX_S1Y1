\section{Giới thiệu}
\paragraph{Đây là bài báo cáo Đồ án \#2, môn Thực hành Giới thiệu ngành Trí tuệ Nhân tạo. \\Đồ án \#2 có đề tài: Xây dựng chương trình AI. \\Đồ án \#2 được thực hiện bởi nhóm các thành viên:}

\begin{itemize}
    \item Đinh Đức Anh Khoa (23122001)
    \item Nguyễn Lê Hoàng Trung (23122002)
    \item Nguyễn Đình Hà Dương (23122004)
    \item Đinh Đức Tài (23122013)
\end{itemize}

\section{Phân công nhiệm vụ}

\textbf{Sau đây là bảng phân công nhiệm vụ cho từng thành viên:}

\begin{itemize}
    \item \textbf{Tuần 2:}
\end{itemize}

\begin{table}[H]
\begin{tabular}{|l|l|l|l|l}
\cline{1-4}
Họ và tên & MSSV & Chức vụ & Nhiệm vụ &  \\ \cline{1-4}
Đinh Đức Anh Khoa & 23122001 & \multicolumn{1}{c|}{Nhóm trưởng} & \begin{tabular}[c]{@{}l@{}}
-  Tìm hiểu Loss Function\\
-  Tìm hiểu quá trình tiền xử lý dữ liệu
\end{tabular} &  \\ \cline{1-4}
Nguyễn Đình Hà Dương & 23122002 & Thành viên & \begin{tabular}[c]{@{}l@{}}
- Tìm hiểu hàm kích hoạt (Activation Function)\\ 
- Tìm hiểu cách mô hình dự đoán mẫu
\end{tabular} &  \\ \cline{1-4}
Nguyễn Lê Hoàng Trung & 23122004 & Thành viên & \begin{tabular}[c]{@{}l@{}}
- Tìm hiểu thuật toán tối ưu (Optimizer) \\
\end{tabular} &  \\ \cline{1-4}
Đinh Đức Tài & 23122013 & Thành viên & \begin{tabular}[c]{@{}l@{}}
- Tìm hiểu cách đánh giá mô hình (Metrics)
\\- Soạn slide trình chiếu, viết báo cáo đồ án
\end{tabular} &  \\ \cline{1-4}
\end{tabular}
\end{table}

\begin{itemize}
    \item \textbf{Tuần 3:}
\end{itemize}

\begin{table}[H]
\begin{tabular}{|l|l|l|l|l}
\cline{1-4}
Họ và tên & MSSV & Chức vụ & Nhiệm vụ &  \\ \cline{1-4}
Đinh Đức Anh Khoa & 23122001 & \multicolumn{1}{c|}{Nhóm trưởng} & \begin{tabular}[c]{@{}l@{}}
- Viết phần dữ liệu, phân chia dữ liệu
\\- Viết quá trình lan truyền tiến, lan truyền\\ ngược, hàm Softmax
\\- Quay demo
%\\-
\end{tabular} &  \\ \cline{1-4}
Nguyễn Đình Hà Dương & 23122002 & Thành viên & \begin{tabular}[c]{@{}l@{}}
- Viết kiến trúc neural network
\end{tabular} &  \\ \cline{1-4}
Nguyễn Lê Hoàng Trung & 23122004 & Thành viên & \begin{tabular}[c]{@{}l@{}}
- Viết về optimizer SGD, quá trình hội tụ, \\kiểm thử mô hình
\\- Chỉnh sửa và tải lên video demo
\end{tabular} &  \\ \cline{1-4}
Đinh Đức Tài & 23122013 & Thành viên & \begin{tabular}[c]{@{}l@{}}
- Viết các tham số của neural network
\\- Soạn slide trình chiếu, viết báo cáo đồ án
\end{tabular} &  \\ \cline{1-4}
\end{tabular}
\end{table}